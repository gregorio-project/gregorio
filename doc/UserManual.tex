\documentclass[12pt]{article}
\usepackage[utf8]{inputenc}
\usepackage[T1]{fontenc}
\usepackage{lmodern}
\usepackage{enumitem}
\usepackage{xspace}
\usepackage{multicol}

\usepackage{makeidx}
\makeindex

\usepackage[table]{xcolor}
\definecolor{lightgray}{gray}{0.9}
\definecolor{green}{HTML}{0c700c}
\definecolor{myred}{HTML}{FF3333}

\usepackage{minted} % must be after xcolor
\newminted{latex}{gobble=2,bgcolor=lightgray}

\usepackage{hyperref}
\hypersetup{colorlinks, citecolor=black, filecolor=black, linkcolor=green, urlcolor=green}

\newcommand*{\eg}{e.g.\@\xspace}
\newcommand*{\nb}{n.b.\@\xspace}
\newcommand*{\ie}{i.e.\@\xspace}

\newcommand{\macroname}[2]{\vspace{3.25ex plus 1ex minus .2ex}%
                           \makebox[\linewidth]{\ttfamily\bfseries #1%
                             \hspace{\fill}\normalfont\itshape #2}%
                         \vspace{1.5ex plus .2ex}%
                         \index{#1}}
                       %\addcontentsline{toc}{subsubsection}{#1}}


\newenvironment{argtable}{\bigskip\rowcolors{1}{lightgray}{lightgray}
                           \begin{tabular}{clp{10cm plus .5cm}}
                             Arg & Value & Description \\
                             \hline}%
                          {\end{tabular}\bigskip}

\setlength{\parindent}{0mm} % Default is 15pt

\begin{document}

\begin{titlepage}
  \begin{center}
    \Huge
    \textcolor{myred}{Gregorio} and \textcolor{myred}{gregoriotex}:

    A project which provides tools for gregorian score engraving.

    \vspace{1cm}\large\href{http://home.gna.org/gregorio/}{Homepage}

    Source code available on
    \href{http://github.com/gregorio-project/gregorio}{GitHub}.
  \end{center}
  \vspace{2cm}
  \begin{multicols}{2}
    \tableofcontents
  \end{multicols}
\end{titlepage}

\section{Gregoriotex Macros}
The following sections document the macros available in the gregoriotex package. The format is as follows:

\macroname{Macroname}{source file}

\begin{argtable}
  Arg \# & Data type & Descriptor\\
\end{argtable}

The source file where the macro is defined is included for developers
wno wish to consult it.\\
Some of the macros intended for inclusion in the main.tex file by user
include usage examples.

\section{User Controls}

These functions are available to the user to customize elements of the score which cannot be controlled from the gabc file.

\subsection{Colors}

Colors are strictly a \LaTeX\ phenomena as currently implemented.  All commands which change the color of text simply print the text without alteration in PlainTeX.

All colors can be redefined using \verb=\definecolor=.  See \verb=xcolor= package for documentation.

\verb=grebackgroundcolor=%tex.sty
?? color behind staves?
\verb={RGB}{255,255,255}= (white)

\verb=gregoriocolor=%tex.sty
Color of elements formatted by \verb=\colored=
\verb={RGB}{229,53,44} (red similar to what is found in liturgical documents)


\subsection{Environments}

Environments are used to apply standard formatting to elements of the score.  Redefining the environment (via a \verb=\renewenvironment= command) allows the user to change how these elements appear in the score.

While environments are technically a \LaTeX\ phenomena, for users of PlainTeX each environment has a pseudo-environment equivalent which is used in place of the environment.  The pseudo-enviroments are called with a \verb=\start*= and \verb=\stop*= command (where ``*'' is the name of the environment).  To change the format of the text in the pseudo-environment, simply redefine these commands (usually you only need to redefine the \verb=\start*= command).  Just make sure that any new definition for \verb=\start*= has \verb=\begingroup= as the first line and for \verb=\stop*= has \verb=\endgroup= as the last line.

\verb=initialformat=%tex.sty, PlainTeX version in tex.tex
	Defines how the first letter of a score appears when using leading initial.  Defaults to \verb=huge= (New Century Roman at 40pt in PlainTeX).
	Deprecated version: \verb=\greinitialformat=

\verb=biginitialformat=%tex.sty, PlainTeX version in tex.tex
	Defines how the first letter of a score appears when using a 2-line leading initial.  Defaults to \verb=Huge= (New Century Roman at 80pt in PlainTeX)
	Deprecated version: \verb=\grebiginitialformat=

\verb=abovelinetextstyle=%tex.sty, PlainTeX version in tex.tex
	Defines how the text placed above the staff lines appears.  Defaults to \verb=small= and \verb=italic= (\verb=italic= only in PlainTeX).
	Deprecated version: \verb=\greabovelinestextstyle=

\verb=translationformat=%tex.sty, PlainTeX version in tex.tex
	Defines how the translation text appears.  Defaults to \verb=italic=.

\verb=normalstafflinesformat=%tex.sty, PlainTeX version in tex.tex
	Defines how the staff lines will appear.  Note that the staff lines are drawn with \verb=\hrule= and so very few things will actually affect their appearance (color being the most prominent one that does).  Empty default.
	Deprecated version: \verb=\grenormalstafflinesformat

\verb=additionalstafflinesformat=%tex.sty, PlainTeX version in tex.tex
	Defines how the additional staff lines (the little ones which are drawn for notes above or below the normal staff lines) appear.  Defaults to \verb=normalstafflinesformat=.
	Deprecated version: \verb=\greadditionalstafflinesformat

\verb=lowchoralsignstyle=%tex.sty, PlainTeX version in tex-signs.tex
	Defines how the low choral signs should appear.  Empty default.
	Deprecated version: \verb=\grelowchoralsignstyle=

\verb=highchoralsignstyle=%tex.sty, PlainTeX version in tex-signs.tex
	Defines how the high choral signs should appear.  Empty default.
	Deprecated version: \verb=\grelowchoralsignstyle=

\subsection{Commands}

In general, commands should not be modified.  Exceptions are noted below.

\verb=\colored#1=%tex.sty
	Colors its argument with \verb=gregoriocolor=
	Modify color by changing \verb=gregoriocolor=.

\verb=\coloredlines#1=%tex.sty
	Colors the staff lines.
	#1 must be a named color defined using \verb=\definecolor=
	Deprecated version: \verb=\grecoloredlines=

\verb=\redlines=%tex.sty
	Short cut for coloring the staff lines \verb=gregoriocolor=.  Equivalent to \verb=\coloredlines{gregoriocolor}=
	Modify color by changing \verb=gregoriocolor=.
	Deprecated version: \verb=\greredlines=

\verb=\normallines=%tex.sty
	Removes all formatting from staff lines.
	Deprecated version: \verb=\grenormallines=

\verb=\setstaffsize#1=%tex.tex
	Changes the size of the staff (and the neumes placed on the staves).  #1 is an integer. Larger values for larger staves, smaller values for smaller staves.  Default value is 17.  \emph{Note:} This does not change the size of the accompanying text (lyrics and/or translations).
	Deprecated version: \verb=\setgrefactor=

\verb=\addtranslationspace=%Not sure this needs to be a user function%tex.tex
	Add space below the stave for a translation.  Has no effect if space is already allocated for a translation.

\verb=\removetranslationspace=%Not sure this needs to be a user function%tex.tex
	Removes the space below the stave for the translation (un-does \verb=\addtranslationspace=).  Has no effect if space is currently not allocated to a translation.

\verb=\setfirstannotation#1=%tex.tex
	Macro to set the first (top) annotation above the initial.  Argument is string (the contents of the annotation).  This macro automatically aligns the top of the annotation with the 4th bar line.  This is equivalent to \verb=\setfirstlineaboveinitial{#1}{#1}=.
	Deprecated version: \verb=\gresetfirstannotation=
	Deprecated version: \verb=\writemode= (\nb this applied small caps and bold automatically while \verb=\setfirstannotation= does not.)%tex.sty, PlainTeX version in tex.tex

\verb=\setfirstlineaboveinitial#1#2=%tex.tex
	Macro to set the first (top} annotation above the initial.  This macros allows you to control how far below the top of the staff #1 (string containing contents of annotation) appears via the second argument.  When its height is 0, the baseline of the annotation aligns with the 4th bar line.  Positive values for the height push the annotation down.  The argument cannot have a negative height.  #2 can be pretty much anything with a visible representation.  \verb=\newline= and \verb=\\= are not respected in either #1 or #2.  Note: large annotations which stick out above the staff will push the commentary up.
	Deprecated version: \verb=\gresetfirstlineaboveinitial=

\verb=\setsecondannotation#1=%tex.tex
	Macro to set the second (bottom) annotation above the initial.  Argument is string containing contents of annotation.
	Deprecated version: \verb=\gresetsecondannotation=

\verb=\scorereference=%tex.tex
	Does nothing.

\verb=\commentary=%tex.tex
	Marco to place the commentary (usually the scriptural reference) in the top right-hand corner of the score.  While individual calls do not support multiple lines, the macro can be called multiple times; each call will typeset a new line.
	
\verb=\removelines=%tex.tex
	Macro to remove the staff lines.
	Deprecated version: \verb=\greremovelines=

\verb=\donotremovelines=%tex.tex
	Macro to force staff lines (undoes the effects of \verb=\removelines=).
	Deprecated version: \verb=\gredonotremovelines=

\verb=\settranslationcenteringscheme#1=%tex.tex
	Macro to change the centering scheme for the translation.
	#1 --- integer: 0=translation is left aligned with the corresponding text; 1=translation is centered below the corresponding text
	Deprecated version: \verb=\setgretranslationcenteringscheme=

\verb=\setnlbintranslation#1=%tex.tex
	Macro to change whether line breaks are allowed in the translations.
	#1 --- integer: 0=line breaks are allowed; 1=line breaks are prohibited

\verb=\blockcustos=%tex.tex
	Macro to block custom.  Applies to all subsequent scores in group.
	Deprecated version: \verb=\greblockcustos=

\verb=\GreSetStaffLinesFormat#1=%tex.tex
	Deprecated.  Used to set the format for the staff lines.  See above notes on the environments for updated documentation.

\verb=\includescore#1=%tex.tex
	Macro for including scores.  Works on both gabc and tex files.  #1 is string specifying the filename and location of the score (relative or absolute).

\verb=\includetexscore#1=%tex.tex
	Macro for including scores which have already been run through gregorio (i.e. are in gtex format).  #1 is string specifying the filename and location of the score (relative or absolute).
	Deprecated version: \verb=\greincludetexscore=

\verb=\includegabcscore#1=%tex.tex
	Macro for including scores which are in gabc format.  This macro will run gregorio on the file before including it.  #1 is string specifying the filename and location of the score (relative or absolute).
	Deprecated version: \verb=\greincludegabcscore=

\verb=\GreUseNormalHyphen=%tex.tex
	Tell gregoriotex to use normal hyphens at the end of lines (default behavior).
	Note: Placing an explicit hyphen in between special character brackets (\ie <sp>-</sp>) in the gabc file will always result in a zero-width hyphen.
	Note: This command only affects the hyphens within a score.  Hyphens elsewhere in a document are unaffected (hence the "Gre" in the command name).
	
\verb=\GreUseZeroHyphen=%tex.tex
	Tell gregoriotex to use zero-width hyphens at the end of lines.
	Note: This command only affects the hyphens within a score.  Hyphens elsewhere in a document are unaffected (hence the "Gre" in the command name).

\verb=\GreForceHyphen=%tex.tex
	Tell gregoriotex to force the appearance of hyphens between all syllables.
	Note: This command only affects the hyphens within a score.  Hyphens elsewhere in a document are unaffected (hence the "Gre" in the command name).

\verb=\removeclef=%tex-signs.tex
	Macro to remove the clefs from a score.
	Deprecated version: \verb=\greremoveclef=

\verb=\normalclef=%tex-signs.tex
	Macro to restore clefs to a score (undoes the effects of \verb=\removeclef=).
	Deprecated version: \verb=\grenormalclef=

\verb=\removecusto=%tex-signs.tex
	Macro to empty the box containing the custos.
	%I’m not sure what the purpose of this macro is.  Need to figure out where it is used.


\section{Gregoriotex Controls}

These functions are the ones used by Gregoriotex internally as it process the commands listed above.  They should not appear in any user document and are listed here for programmer documentation purposes only.

\verb=\gre@coloredlines#1=%tex.sty
	Changes the color of the staff lines to \#1.  \#1 must be a named color defined using \verb=\definecolor=

\verb=\gre@redlines=%tex.sty
	Changes the color of the staff lines to \verb=gregoriocolor=.

\verb=\gre@normallines=%tex.sty
	Resets the formatting of the staff lines.

\verb=\gre@error#1=%tex.tex
	Raises an error which is identified as coming from GregorioTeX.  Uses \LaTeX \verb=\PackageError= or PlainTeX \verb=\errmessage= as appropriate.  \#1 (a string) is the accompanying message.
	
\verb=\gre@warn#1=%tex.tex
	Raises a warning which is identified as coming from GregorioTeX.  Uses \LaTeX \verb=\PackageWarning= or PlainTeX \verb=\message= as appropriate.  <string is the accompanying message.
	
\verb=\gre@localleftbox=%tex.tex
	Alias for \verb=\luatexlocalleftbox=.  Used to make propagating changes in latex easier.

\verb=\gre@localrightbox=%tex.tex
	Alias for \verb=\luatexlocalrightbox=.  Used to make propagating changes in latex easier.
	
\verb=\gre@unsetattribute=%tex.tex
	Alias for \verb=\unsetlutexattribute{\gregorioattr}=

\verb=\gregorioattr=%tex.tex
	A luatex attribute we put on the text nodes.
	If it is 1, it means that there may be a dash here if this syllable is at the end of a line.
	If it is 2, it means that it's never useful to typeset a dash.
	If it is 0, it just means that we are in a score.

\verb=\gregoriocenterattr=%tex.tex
	A luatex attribute used for translation centering.

\verb=\gregoriotexversion#1=%tex.tex
	Defines the current version of GregorioTeX API.  \#1 is an integer date in format: yyyymmdd

\verb=\gre@internalversion#1=%tex.tex
	The version of GregorioTeX.  \#1 is an integer date in format: yyyymmdd

\verb=\gre@declarefileversion#1#2=%tex.tex
	Checks to see if GregorioTeX component \#1 (a string containing the name) with version \#2 (an integer in date format yyyymmdd) is compatible with \verb=\gre@internalversion=

\verb=\gre@factor=%tex.tex
	Count representing the size of the staff.  Initialized to 0, but changed to 17 by \verb=\begingregorianscore= if user hasn't changed it.
	
\verb=\gre@stafflinewidth=%tex.tex
	Dimension representing the width of a line of staff.  Can vary, for example, at the first line.

\verb=\gre@linewidth=%tex.tex
	Dimension representing the width of the score (including initial).

\verb=\gre@calculateconstantglyphraise=%tex.tex
	Macro to caluclate \verb=\gre@constantglyphraise=

\verb=\gre@constantglyphraise=%tex.tex
	Dimension representing ??

\verb=\gre@currenttranslationheight=%tex.tex
	Dimension representing the space for the translation beneath the text.

\verb=\gre@addtranslationspace=%tex.tex
	Macro to tell Gregorio to set space for the translation.

\verb=\gre@removetranslationspace=%tex.tex
	Macro to tell Gregorio to remove the space allocated to the translation.

\verb=\gre@kernbeforeeol=%tex.tex
	Macro describing a kern to make before ending the line, which we sometimes want (see \verb=\gresyllable=)

\verb=\gre@newlinecommon#1#2#3#4#5=%tex.tex
	Macro we call each time we force a changing of line, it automatically sets \verb=\greknownline=, and adjusts left spaces.
	\#1 --- integer: 0 = no note above staff; 1 = note above 4th line (position k); 2 = note on 5th line (position l); 3 = note above 5th line (position m)
	\#2 --- integer: 0 = no note below staff; 1 note below 1st line (position c); 2 = note on 0th line (position b); 3 = note below 0th line (position a); 4 = note below 0th line (position a) with vertical episemus
	\#3 --- integer: 0 = no translation; 1 = translation present
	\#4 --- integer: 0 = justify line; 1 = do not justify line
	\#5 --- integer: 0 = no extra space above staff; 1 = extra space above staff

\verb=\gre@additionalbottomspace=%tex.tex
	Dimension representing extra space below the staff needed for low notes.

\verb=\gre@additionaltopspace=%tex.tex
	Dimension representing extra space above the staff needed for high notes.

\verb=\gre@updateadditionalspaces#1#2=%tex.tex
	Macro which updates \verb=\gre@additionalbottomspace= and \verb=\gre@additionaltopspace=
%	\#1 --- integer: 0 = no note above staff; 1 = note above 4th line (position k); 2 = note on 5th line (position l); 3 = note above 5th line (position m)
%	\#2 --- integer: 0 = no note below staff; 1 note below 1st line (position c); 2 = note on 0th line (position b); 3 = note below 0th line (position a); 4 = note below 0th line (position a) with vertical episemus

\verb=\gre@textlower=%tex.tex
	Dimension representing the height of the separation between the 0th line (which is invisible except for notes in the a or b position) and the bottom of the text.

\verb=\gre@Tempwidth=%tex.tex
	Box used to calculate \verb=\gre@tempwidth=
	
\verb=\gre@tempwidth=%tex.tex
	Dimension representing width of some element.

\verb=\gre@widthof#1=%tex.tex
	Macro which calculates \verb=\gre@tempwidth= as width of \#1.

\verb=\gre@textaligncenter=%tex.tex
	Dimension representing the width from the beginning of the letters in a syllable to the middle of the middle letters.  Used for lining up neumes and syllables.
	
\verb=\gre@findtextaligncenter#1#2#3=%tex.tex
	Macro for calculating \verb=\gre@textaligncenter=.
	\#1 --- string: The first part of the syllable (any preceding consonants in Latin)
	\#2 --- string: The middle part of the syllable (the vowel in Latin)
	\#3 --- integer: syllable the calculation is being performed for; 0 = current syllable; 1 = next syllable

\verb=\gre@additionalleftspace=%tex.tex
	Dimension representing the additional space that has to be added to the localleftbox for a big initial (one taking two lines).

\verb=\Gre@Initial=%tex.tex
	Box containing the initial.

\verb=\gre@initialwidth=%tex.tex
	Dimension representing the width of the initial (and the space after).

\verb=\gre@biginitial=%tex.tex
	Count indicating whether initial takes 2 lines: 0 = it doesn't; 1 = it does.

\verb=\gre@knowline=%tex.tex
	Line gregoriotex thinks is being set.

\verb=\gre@updateleftbox=%tex.tex
	Macro which adjusts size of line and clef placement after initial.

\verb=\gre@updatelinewidth=%tex.tex
	Macro which adjusts the width of the line.

\verb=\gre@initialformat#1=%tex.tex
	Macro which applies formatting in \verb=initialformat= environment to initial (\#1, a character).

\verb=\gre@biginitialformat#1=%tex.tex
	Macro which applies formatting in \verb=biginitialformat= environment to initial (\#1, a character).

\verb=\Gre@Aboveinitialfirstbox=%tex.tex
	The box which contains the first (top) annotation above the initial.

\verb=\gre@aboveinitialfirstraise=%tex.tex
	Dimension representing the space allocated to the first annotation.

\verb=\Gre@Aboveinitialsecondbox=%tex.tex
	The box which contains the second (bottom) annotation above the initial.

\verb=\gre@aboveinitialsecondraise=%tex.tex
	Dimension representing the space allocated to the second annotation.

\verb=\gre@setaboveinitialrais=%tex.tex
	Macro to give \verb=\gre@aboveinitialfirstraise= and \verb=\gre@aboveinitialsecondraise= their working values.
	
\verb=\gre@currentabovelinestextheight=%tex.tex
	Dimension representing the space allocated above the lines for text.	

\verb=\gre@abovelinestextstyle#1=%tex.tex
	Macro to apply the above line text style to argument (a character).

\verb=\gre@addspaceabove=%tex.tex
	Macro to allocate space for text placed above the staff lines.

\verb=\gre@removespaceabove=%tex.tex
	Macro to unallocate space for text above the staff lines.

\verb=\gre@currenttextabovelines#1=%tex.tex
	Macro containing the text (\#1) which is currently being placed above the staff lines.

\verb=\gre@typsettextabovelines#1=%tex.tex
	Macro to typeset argument (a string) above the staff lines.

\verb=\gre@removelinescount=%tex.tex
	Boolean indicating whether staff lines should be printed.  Has value 0 when lines are printed, 1 when they are not.

\verb=\gre@drawfirstlines=%tex.tex
	Macro to draw the staff lines for the first line of the score (\ie it accounts for the space taken up by the initial).

\verb=\Gre@Lines=%tex.tex
	Box to contain the staff lines for lines other than the first line.

\verb=\gre@generatelines=%tex.tex
	Macro to fill \verb=\Gre@Lines=.

\verb=\gre@smallsecondline=%tex.tex
	Macro called when the initial is big to make the second set of staff lines the same length as the first.

\verb=\gre@normallines=%tex.tex
	Marco called after the second set of staff lines when the initial is big to go back to normal width lines.

\verb=\gre@translationcenteringscheme=%tex.tex
	Boolean to indicate the centering scheme used for the translation.
	Possible values: 0=translation is left aligned with corresponding text; 1=translation is centered with the corresponding text

\verb=\gre@nlbintranslation=%tex.tex
	Variable used to indicate whether line breaks are allowed in the translation.
	Possible values: 0=line breaks are allowed; 1=line breaks are prohibited

\verb=\gre@translationformat#1=%tex.tex (is this defined somewhere else too?)
	Macro to apply the translation format to string.

\verb=\gre@mustdotranslationcenterend=%tex.tex
	Boolean to indicate if the translation is at the end of a line?
	Possible values: 0=not at the end of a line; 1=at the end of a line

\verb=\gre@dotranslationcenterend=%tex.tex
	Macro to set things up for a translation at the end of a line.

\verb=\gre@printchirovbars=%tex.tex
	Count that is 1 or 0 if we need to print the small vertical bars in the chironomic line.

\verb=\gre@nolastline=%tex.tex
	Macro to call when there is just a little thing that will go to the last line, when it is not necessary
	It doesn't seem to be used, so it's a good candidate for deprecation!

\verb=\gre@endofword#1=%tex.tex
	Macro called at the end of each word.  Does extra stuff when argument (an integer) is 1?

\verb=\gre@endbeforebar#1=%tex.tex
	Macro called at end of word when next element is a bar.  Argument (an integer) has same function as in \verb=\gre@endofword=.

\verb=\gre@endafterbar#1=%tex.tex
	Macro called after a bar.  Argument (an integer) has same function as in \verb=\gre@endofword=.

\verb=\gre@lastoflinecount=%tex.tex
	Count which marks the last syllable of the line.
	Possible values: 0=nothing; 1=last syllable of line; 2=first syllable of line

\verb=\gre@disableeolshifts=%tex.tex
	Boolean which indicates that the syllable should be shifted left a bit.
	Possible values: 0=shift happens; 1=shift doesn't happen.

\verb=\Gre@DisableEOLShifts=%tex.tex
	Macro to set \verb=\gre@disableeolshifts= to 1.

\verb=\Gre@EnableEOLShifts=%tex.tex
	Macro to set \verb=\gre@disableeolshifts= to 0.

\verb=\gre@blockcusto=%tex.tex
	Count to indicate if the custo should be blocked.
	Possible values: 0=do not block custo; 1=block custo

\verb=\gre@endofsyllable=%tex.tex
	Macro called at the end of a syllable which does not end a word.

\verb=\gre@nlbstate=%tex.tex
	Count to indicate the no line break areas.
	Possible values: 0 = not a no line break area; 1 = no line break due to translation centering; 2 = no line break due to <nlba> tag

\verb=\gregoriofontname=%tex.tex
	Macro which holds the default font name (greciliae).

\verb=\gre@usestylefont=%tex.tex
	Count to indicate if gregoriostylefont should be used.
	Possible values: 0 = do not use; 1 = use

\verb=\gre@setstylefont=%tex.tex
	Macro to load greextra as gregoriostylefont at correct size.
	
\verb=\gre@normalstafflinesformat=%tex.tex
	Macro to apply formatting to the normal staff lines.
	
\verb=\gre@additionalstafflinesformat=%tex.tex
	Macro to apply formatting to the additional staff lines.

\verb=\gre@includetexscore#1=%tex.tex
	Macro to include scores in gtex format.

\verb=\gre@includegabcscore#1=%tex.tex
	Macro to include scores in gabc format.

\verb=\gre@usestylecommon=%tex-signs.tex
	Checks to see if gregoriostylefont has been loaded and loads it if it has not.

\verb=\gre@falsepenalty#1=%tex-signs.tex
	Does nothing.

\verb=\gre@truepenalty#1=%tex-signs.tex
	Alias for \verb=\penalty#1=.  Used in combination with \verb=\gre@falsepenalty= to avoid placing \verb=\penalty= where it isn't allowed.

\verb=\gre@hskip=%tex-signs.tex
	Alias for \verb=\hskip= outside of discretionaries and \verb=\kern= inside of them.

\verb=\gre@penalty#1=%tex-signs.tex
	Alias for \verb=\gre@truepenalty= outside of discretionaries and \verb=\gre@falsepenalty= inside of them.

\verb=\gre@insidediscretionary=%tex-signs.tex
	Count for tracking whether we are in a discretionary (1) or not (0).

\verb=\gre@discretionary#1#2=%tex-signs.tex
	Sets \verb=\gre@hskip=, \verb=\gre@penalty=, \verb=\gre@insidediscretionary= to their within discretionary values and then calls \verb=\discretionary#1#2#3=.  \#2 (a string) is a non-printing string which sets \verb=\gre@lastoflinecount= to 2.  After resolving the \verb=\discretionary= command \verb=\gre@hskip=, \verb=\gre@penalty=, and \verb=\gre@insidediscretionary= are restored to their outside values.

\verb=\gre@removeclefcount=%tex-signs.tex
	Count for indicating if the clef should be printed (1) or not (0).

\verb=\gre@clefnum=%tex-signs.tex
	Count indicated the clef line and pitch: 1- c on bottom line, 2 - c on second line, 3 - c on third line, 4 - c on top line, 5 - f on bottom line, 6 - f on second line, 7 - f on third line, 8 - f on top line.

\verb=\gre@clefwidth=%tex-signs.tex
	Width of the clef.

\verb=\gre@setlinesclef#1#2#3#4=%tex-signs.tex
	Macro to define the clef that will appear at the beginning of the lines.
	\#1 --- character: type of clef (c or f)
	\#2 --- character: height of clef (using gabc height letters)
	\#3 --- integer: add a space after the clef (1), or not (0)
	\#4 --- character: height of flat after clef (using gabc height letters, a = no flat)

\verb=\gre@calculateclefnum#1#2#3=%tex-signs.tex
	Macro to calculate the clef number and store it in \verb=\gre@clefnum=.
	\#1 --- character: type of clef (c or f)
	\#2 --- integer: line on which the clef appears (1--4)
	\#3 --- character: height of flat after clef (using gabc height letters, a = no flat)

\verb=\gre@updatelinesclef=%tex-signs.tex
	Macro redrawing a key from \verb=\gre@clefnum=.  Useful for vertical space changes.
	
\verb=\gre@typekey#1#2#3#4#5=%tex-signs.tex
	Macro for typesetting the key signature.
	\#1 --- character: type of clef (c or f)
	\#2 --- integer: line of the key (1 to 4)
	\#3 --- integer: Use small key characters (1), or not (0)
	\#4 --- integer: Print a space after the key (1), or not (0)
	\#5 --- character: Height of flat in key (‘a’ for no flat)

\verb=\gre@inchangeclef#1#2#3=%tex-signs.tex
	Macro for changing key inside a syllable.
	\#1 --- character: type of clef (c or f)
	\#2 --- integer: line on which the clef appears (1--4)
	\#3 --- character: height of flat after clef (using gabc height letters, a = no flat)

\verb=\gre@setcusto#1=%tex-signs.tex
	Macro for typesetting custos.
	\#1 --- character: height of custom

\verb=\gre@additionaltopcusotolineend=%tex-signs.tex
	Macro for typesetting the line behind a high custos at end of line.

\verb=\gre@additionalbottomcusotlineend=%tex-signs.tex
	Macro for typesetting the line behind a low custos at end of line.

\verb=\gre@additionaltopcusotolinemiddle=%tex-signs.tex
	Macro for typesetting the line behind a high custos in middle of line.

\verb=\gre@additionalbottomcusotlinemiddle=%tex-signs.tex
	Macro for typesetting the line behind a low custos in middle of line.

\verb=\gre@custocahar#1=%tex-signs.tex
	Selects the appropriate custos character.
	\#1 --- character: height of the custos

\verb=\gre@curlybracechar=%tex-signs.tex
	Macro for the char ‘\{‘ (which is rotated 90 and placed over the notes)

\verb=\gre@bracechar=%tex-signs.tex
	Macro for the char ‘(‘ (which is rotated 90 and placed over the notes)

\verb=\gre@resizebox=%tex-signs.tex
	Local alias for \verb=\resizebox= (currently borrowed from graphics)

\verb=\gre@lowchoralsignstyle#1=%tex.sty and tex-signs.tex
	Applies the formatting to the low choral signs specified by the appropriate (pseudo-)environment.

\verb=\gre@highchoralsignstyle#1=%tex.sty and tex-signs.tex
	Applies the formatting to the high choral signs specified by the appropriate (pseudo-)environment.

\verb=\Gre@Tempsign=%tex-signs.tex
	Box for temporarily holding a sign needed for calculations.

\verb=\gre@tempdimsignwidth=%tex-signs.tex
	Temporary dimension used to calculate placement of signs.

\verb=\gre@vepisemusorrareaux#1#2#3#4#5#6#7=%tex-signs.tex
	Macro help with typesetting of episemus.
	\#1 --- character: A preceding symbol needed for calculating position of episemus.
	\#2 --- character: Another preceding symbol.
	\#3 --- integer: Indicates how far to move back; 0 = go back to beginning of glyph, 1 = go back width of \#2, 2 = go back width of \#1 and forward width of \#2, 3 = go to beginning of glyph, forward by width of \#1 and back width of \#2
	\#4 --- integer: An additional shift (in sp) to apply to the episemus.
	\#5 --- integer: Use ictus arsicus (1) or ictus theticus (2)
	\#6 --- integer: Set a vertical episemus (1), a rare sign (2), a choral sign (3), or a brace above the bar (4)
	\#7 --- integer: If \#6=3 then the choral sign.

\verb=\gre@vepisemusorrare#1#2#3#4#5=%tex-signs.tex
	Macro for typesetting vertical episemus or rare accent.
	\#1 --- character: height of episemus
	\#2 --- integer: Type of glyph the episemus is attached to.  See episemus special argument for description of options.
	\#3 --- integer: Glyph number of glyph being typeset
	\#4 --- integer: Set a vertical episemus (1), a rare sign (2), a choral sign (3), or a brace above the bar (4)
	\#5 --- string: If \#4=3 then the choral sign.

\verb=\grewriteaux#1=%tex.tex
	Macro for writing to the aux file.
	\#1 --- string: what is written to the aux file.

\verb=\opengreaux=%tex.tex
	Macro for opening the aux file.

\verb=\closegreaux=%tex.tex
	Macro for closing the aux file.

\verb=\gre@ictus#1=%tex-signs.tex
	Macro for writing the position of an ictus to the aux file.
	\#1 --- integer: type of ictus which was typeset, 1 = type a, 2 = type t

\verb=\gre@hepisorlineaux#1#2#3#4=%tex-signs.tex
	Macro for helping to typeset a horizontal episemus and additional line.
	\#1 --- character: a glyph having the same width as the width between the end of the glyph and the beginning of the episemus
	\#2 --- integer: glyph number of the episemus
	\#3 --- integer: go to beginning of glyph (0), or not (else)
	\#4 --- integer: set horizontal episemus (0), horizontal episemus under a note (1), line at top of staff (2), line at bottom of staff (3), choral sign (4)

\verb=\gre@hepisorline#1#2#3#4=%tex-signs.tex
	Macro for typesetting a horizontal line (additional line or episemus).
	\#1 --- character: Height of the episemus
	\#2 --- special: See Episemus special
	\#3 --- integer: The ambitus of the porrectus or porrectus flexes if the second argument is 9, 10, 11, 21, 22, 23; ignored otherwise
	\#4 --- integer: set horizontal episemus (0), horizontal episemus under a note (1), line at top of staff (2), line at bottom of staff (3), choral sign (4)

\verb=\gre@AddHEpisemusBridges=%tex-signs.tex
	Macro to activate bridging horizontal episemus.
	
\verb=\gre@RemoveHEpisemusBridges=%tex-signs.tex
	Macro to suspend bridging horizontal episemus.

\verb=\gre@writebar#1#2#3=%tex-signs.tex
	Macro to typeset a bar line.
	\#1 --- integer: Type of bar line: virgula (0), minima (2), maior (3), finalis (4), last finalis (5), dominican bar 1 (6), dominican bar 2 (7), dominican bar 3 (8), dominican bar 4 (9), dominican bar 5 (10), or dominican bar 6 (11)
	\#2 --- integer: The bar appears within a syllable (1), or not (0)
	\#3 --- code: Macros which may happen before the skip but after the bar (typically \verb=\grevepisemus=)

\verb=\gre@tempdimtwo=
	A temporary dimension used in calculations.

\verb=\gre@divisiomaiorsymbol=%tex-signs.tex
	Macro defining the divisio maior symbol.

\verb=\gre@divisiofinalissymbol=%tex-signs.tex
	Macro defining the divisio finalis symbol.

\verb=\keeprightbox=%tex-signs.tex
	A count to tell if we have to keep the localrightbox until the end.

\verb=\gre@hidepclines=%tex-signs.tex
	An integer indicating whether the lines behind a punctum cavum are hidden (1) or not (0).

\verb=\gre@hidealtlines=%tex-signs.tex
	An integer indicating whether the lines behind an alteration are hidden (1) or not (0).

\verb=\gre@fillhole#1=%tex-signs.tex
	Character to fill a hole in a character with the color grebackgroundcolor.
	\#1 --- character: The shape of the hole.

\verb=\gre@firstisalteration=%tex-signs.tex
	Count to track if the first glyph is an alteration.

\verb=\gre@alteration#1#2#3#4=%tex-signs.tex
	Macro to typeset an alteration (sharp, flat, or natural).
	\#1 --- character: The height of the alteration
	\#2 --- character: The alteration itself.
	\#3 --- character: The character of the hole for the alteration.
	\#4 --- integer: Indicates a flat for a key change (1), or not (0).

\section{Special arguments}

These arguments are used by multiple functions and take a lot of space to describe so we describe them once here and refer to this section rather than have multiple definitions.

\subsection{Episemus special}
Integer with the following possibilities:
    0: last note, which is a standard punctum (works with pes)
    1: same, but the last note is a deminutus
    2: the note before the last note, which is a standard punctum
    3: idem, but the note is the note preceding a deminutus
    4: the note before the note before the last note (for porrectus flexus)
    5: idem, but when the two last notes are a deminutus
    6: the first note, if it is a standard punctum
    7: the first note, if it is an initio debilis
    8: the first note, if it is a porrectus
    % the three next arguments make no sense for a vepisemus
    9: the two first notes, if it is a porrectus
    10: the two first notes, if it is a porrectus flexus
    11: the notes two and three of a torculus resupinus
    12: the last note, if it is a punctum inclinatum
    13: idem, if it is a punctum inclinatum deminutus
    14: idem, if it is a stropha
    15: idem, with a quilisma
    16: idem, with an oriscus
    17: same of 2 but for ambitus of one
    18: same of 0, but the last note is a smaller punctum (concerning simple podatus, podatus, and torculus resupinus)
    19: the first note, if it is an oriscus
    20: the first note, if it is a quilisma
    21: the second note of a torculus resupinus with first ambitus of at least two
    22: idem with ambitus of one
    23: idem with initio debilis
    24: the last note, if it is a linea punctum (or linea punctum cavum)
    25: the last note, if it is a bar
    26: the last note, if it is a virgula
    27: the last note, if it is a divisio finalis


%%% Local Variables:
%%% mode: latex
%%% TeX-master: "Documentation"
%%% End:

% !TEX root = GregorioRef.tex
% !TEX program = LuaLaTeX+se
\section{Gregorio Controls}

These functions are the ones written by Gregorio to the gtex file.
While one could, in theory, use/change them to alter the appearance of
elements of the score, it is far better to make your changes in the
gabc file and let Gregorio make the changes to the gtex file.

\macroname{\textbackslash GreBeginScore}{}{gregoriotex-main.tex}
Macro to start a score.

\macroname{\textbackslash GreEndScore}{}{gregoriotex-main.tex}
Macro to end a score.

\macroname{\textbackslash GreAccentus}{\#1\#2}{gregoriotex-signs.tex}
Macro for typesetting an accentus.

\begin{argtable}
  \#1 & character & height of episemus\\
  \#2 & integer   & Type of glyph the episemus is attached to. See \nameref{EpisemusSpecial} argument for description of options.\\
\end{argtable}

\macroname{\textbackslash GreAdditionalLine}{\#1\#2\#3}{gregoriotex-signs.tex}
Macro to typeset the additional line above or below the staff.

\begin{argtable}
  \#1 & special & See \nameref{EpisemusSpecial}.\\
  \#2 & integer & The ambitus of the porrectus or porrectus flexes if the first argument is 9, 10, 11, 21, 22, 23; ignored otherwise.\\
  \#3 & integer & Set horizontal episemus (0), horizontal episemus under a note (1), line at top of staff (2), line at bottom of staff (3), choral sign (4).\\
\end{argtable}

\macroname{\textbackslash GreAdjustSecondLine}{gregoriotex.tex}
%didn't actually find this one in gregoriotex-write.c, classified it here based on it’s related function GreAdjustThirdLine
Macro to call before first syllable, but after \verb=\GreSetInitialClef=.

\macroname{\textbackslash GreAdjustThirdLine}{gregoriotex-main.tex}
Macro to call during the second line.

\macroname{\textbackslash GreAugmentumDuplex}{\#1\#2\#3}{gregoriotex-signs.tex}
Macro for typesetting an augmentum duplex (a pair of punctum mora)

\begin{argtable}
  \#1 & character & Height for first punctum mora.\\
  \#2 & character & Height for second punctum mora.\\
  \#3 & integer   & First punctum mora occurs before last note of a podatus, prorectus, or toculus resupinus (1), or not (0).\\
\end{argtable}

\macroname{\textbackslash GreBarBrace}{\#1}{gregoriotex-signs.tex}
Macro for typesetting a bar brace.

\begin{argtable}
  \#1 & integer & Type of glyph the episemus is attached to.  See \nameref{EpisemusSpecial} argument for description of options.\\
\end{argtable}

\macroname{\textbackslash GreBarSyllable}{\#1\#2\#3\#4\#5\#6\#7\#8\#9}{gregoriotex-syllable.tex}
Macro for typesetting a bar syllable.

\begin{argtable}
  \#1 & string & First part of syllable text associated with the bar.\\
  \#2 & string & Middle part of the syllable text associated with the bar.\\
  \#3 & string & End part of the syllable text associated with the bar.\\
  \#4 & & Not used.\\
  \#5 & \TeX\ code & Code to execute before calculating \texttt{nextbegindifference}.\\
  \#6 & & Not used.\\
  \#7 & $0 <$ integer $< 19$ & the type of notes alignment.  See \nameref{notesalign}.\\
  & $20 <$ integer $< 39$ & Same as below 20 except there is a flat before the notes.  Subtract 20 to get the type of notes alignment.\\
  & $40 <$ integer $< 59$ & Same as below 20 except there is a natural before the notes.  Subtract 40 to get the type of notes alignment.\\
  \#8 & \TeX\ code & Code to execute before  printing anything.\\
  \#9 & \texttt{GreWriteBar} & The bar line.
\end{argtable}

\macroname{\textbackslash GreBarVEpisemus}{\#1}{gregoriotex-signs.tex}
Macro to typeset a vertical episemus around a bar.

\begin{argtable}
  \#1 & integer & Type of glyph the episemus is attached to.  See \nameref{EpisemusSpecial} argument for description of options.\\
\end{argtable}

\macroname{\textbackslash GreBeginNLBArea}{\#1\#2}{gregoriotex-main.tex}
Macro called at beginning of a no line break area.

\begin{argtable}
  \#1 & 0 & Not in the neumes.\\
      & 1 & In the neumes.\\
  \#2 & 0 & Call didn't come from translation centering.\\
      & 1 & Call came from translation centering.
\end{argtable}




\macroname{\textbackslash grefirstlinebottomspace}{\#1\#2}{gregoriotex-spaces.tex}
Macro for calculating the additional space needed below the first line of the score.

\begin{argtable}
  \#1 & 0 & no notes below the staff\\
  & 1 & note below the first staff line (c position)\\
  & 2 & note on the 0 staff line (b position)\\
  & 3 & note below the 0 staff line (a position)\\
%  & 4 & note below the 0 staff line with a vertical episemus attached\\
  \#2 & 1 & there is a translation below the text\\
  & 0 & there is no translation below the text
\end{argtable}

%%% Local Variables:
%%% mode: latex
%%% TeX-master: "GregorioRef"
%%% End:


\addcontentsline{toc}{section}{Index}
\printindex

\end{document}
