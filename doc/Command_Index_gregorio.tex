\section{Gregorio Controls}

These functions are the ones written by Gregorio to the gtex file.
While one could, in theory, use/change them to alter the appearance of
elements of the score, it is far better to make your changes in the
gabc file and let Gregorio make the changes to the gtex file.

\macroname{\textbackslash begingregorioscore}{gregoriotex.tex}
%\verb=\begingregorioscore=%tex.tex
Macro to start a score.

\macroname{\textbackslash endgregorioscore}{gregoriotex.tex}
%\verb=\endgregorioscore=%tex.tex
Macro to end a score.

\macroname{\textbackslash greaccentus\#1\#2}{gregoriotex-signs.tex}
%\verb=\greaccentus#1#2=%tex-signs.tex
Macro for typesetting an accentus.

\begin{argtable}
  \#1 & character & height of episemus\\
  \#2 & integer   & Type of glyph the episemus is attached to. See \nameref{EpisemusSpecial} argument for description of options.\\
\end{argtable}

\macroname{\textbackslash greacticeatechironomy}{gregoriotex.tex}
%\verb=\greactiveatechironomy=%tex.tex
Macro called at the beginning of the score to enable chironomy.

\macroname{\textbackslash greadditionalline\#1\#2\#3}{gregoriotex-signs.tex}
%\verb=\greadditionalline#1#2#3=%tex-signs.tex
Macro to typeset the additional line above or below the staff.

\begin{argtable}
  \#1 & special & See \nameref{EpisemusSpecial}.\\
  \#2 & integer & The ambitus of the porrectus or porrectus flexes if the first argument is 9, 10, 11, 21, 22, 23; ignored otherwise.\\
  \#3 & integer & Set horizontal episemus (0), horizontal episemus under a note (1), line at top of staff (2), line at bottom of staff (3), choral sign (4).\\
\end{argtable}

\macroname{\textbackslash greadjustsecondline}{gregoriotex.tex}
%\verb=\greadjustsecondline=%didn't actually find this one in gregoriotex-write.c%tex.tex
Macro to call before first syllable, but after \verb=\gresetinitialclef=.

\macroname{\textbackslash greadjustthirdline}{gregoriotex.tex}
\verb=\greadjustthirdline=%tex.tex
Macro to call during the second line.

\macroname{\textbackslash greaugmentumduplex\#1\#2\#3}{gregoriotex-signs.tex}
%\verb=\greaugmentumduplex#1#2#3=%tex-signs.tex
Macro for typesetting an augmentum duplex (a pair of punctum mora)

\begin{argtable}
  \#1 & character & Height for first punctum mora.\\
  \#2 & character & Height for second punctum mora.\\
  \#3 & integer   & First punctum mora occurs before last note of a podatus, prorectus, or toculus resupinus (1), or not (0).\\
\end{argtable}

\macroname{\textbackslash grebarbrace\#1}{gregoriotex-signs.tex}
%\verb=\grebarbrace#1=%tex-signs.tex
Macro for typesetting a bar brace.

\begin{argtable}
  \#1 & integer & Type of glyph the episemus is attached to.  See \nameref{EpisemusSpecial} argument for description of options.\\
\end{argtable}

\macroname{\textbackslash grebarsyllable}{gregoriotex-syllable.tex}
%\verb=\grebarsyllable=

\macroname{\textbackslash grebarvepisemus\#1}{gregoriotex-signs.tex}
%\verb=\grebarvepisemus#1=%tex-signs.tex
Macro to typeset a vertical episemus around a bar.

\begin{argtable}
  \#1 & integer & Type of glyph the episemus is attached to.  See \nameref{EpisemusSpecial} argument for description of options.\\
\end{argtable}

\macroname{\textbackslash grebarvepisemusictusa\#1}{gregoriotex-signs.tex}
%\verb=\grebarvepisemusictusa#1=%tex-signs.tex
Macro to typeset a vertical episemus and ictus (type a) around a bar.

\begin{argtable}
  \#1 & integer & Type of glyph the episemus is attached to.  See \nameref{EpisemusSpecial} argument for description of options.\\
\end{argtable}

\macroname{\textbackslash grebarvepisemusictust\#1}{gregoriotex-signs.tex}
%\verb=\grebarvepisemusictust#1=%tex-signs.tex
Macro to typeset a vertical episemus and ictus (type t) around a bar.

\begin{argtable}
  \#1 & integer & Type of glyph the episemus is attached to.  See \nameref{EpisemusSpecial} argument for description of options.\\
\end{argtable}

\macroname{\textbackslash grebeginnlbarea\#1\#2}{gregoriotex.tex}
%\verb=\grebeginnlbarea#1#2=%tex.tex
Macro called at beginning of a no line break area.

\begin{argtable}
  \#1 & 0 & Not in the neumes.\\
      & 1 & In the neumes.\\
  \#2 & 0 & Call didn't come from translation centering.\\
      & 1 & Call came from translation centering.
\end{argtable}

\macroname{\textbackslash grebeginnotes}{gregoriotex.tex}
%\verb=\grebeginnotes=%tex.tex
Macro to draw the staff lines.  Comes after the initial but before the clef.

\macroname{\textbackslash greboldfont\#1}{gregoriotex.sty}
%\verb=\greboldfont#1=%tex.sty, PlainTeX version in tex.tex
Makes argument bold.  Accesses \LaTeX\ \verb=\textbf= or Plain\TeX\ \verb=\bf= as appropriate.  Corresponds to ``<b></b>'' tags in gabc.

\begin{argtable}
  \#1 & string & Text to be typeset in bold.\\
\end{argtable}

\macroname{\textbackslash grechangeclef\#1\#2\#3\#4}{gregoriotex-signs.tex}
%\verb=\grechangeclef#1#2#3#4=%tex-signs.tex
Macro called when key changes

\begin{argtable}
  \#1 & character & Type of new clef (c or f).\\
  \#2 & integer   & Line of new clef.\\
  \#3 & 0         & Print space before clef.\\
      & 1         & Do not print space before clef.\\
  \#4 & character & Height of flat in key (`a' for no flat).\\
\end{argtable}

\macroname{\textbackslash grecirculus\#1\#2}{gregoriotex-signs.tex}
%\verb=\grecirculus#1#2=%tex-signs.tex
Macro for typesetting a circulus.

\begin{argtable}
  \#1 & character & Height of circulus.\\
  \#2 & integer   & Type of glyph the circulus is attached to.  See \nameref{EpisemusSpecial} argument for description of options.\\
\end{argtable}

\macroname{\textbackslash grecolored\#1}{gregoriotex.sty}
%\verb=\grecolored#1=%tex.sty, PlainTeX version in tex.tex
Colors argument (a string) in \verb=gregoriocolor.=  Corresponds to ``<c></c>'' tags in gabc.  Does nothing in Plain\TeX

\macroname{\textbackslash grecusto\#1}{gregoriotex-signs.tex}
%\verb=\grecusto#1=
Typesets a custo.

\begin{argtable}
  \#1 & character & Height of custo.\\
\end{argtable}

\macroname{\textbackslash gredagger}{gregoriotex-symbols.tex}
%\verb=\gredagger=
Macro to typeset a dagger (\dag).

\macroname{\textbackslash grediscretionary}{}
%\verb=\grediscretionary{}=

\macroname{\textbackslash gredivisionfinalis\#1}{gregoriotex-signs.tex}
%\verb=\gredivisiofinalis#1=%tex-signs.tex
Macro to typeset a divisio finalis.

\begin{argtable}
  \#1 & code & Macros which may happen before the skip but after the divisio finalis (typically \verb=\grevepisemus=).\\
\end{argtable}

\macroname{\textbackslash gredivisionmaior\#1}{gregoriotex-signs.tex}
%\verb=\gredivisiomaior#1=%tex-signs.tex
Macro to typeset a divisio maior.

\begin{argtable}
  \#1 & code & Macros which may happen before the skip but after the divisio maior (typically \verb=\grevepisemus=).\\
\end{argtable}

\macroname{\textbackslash gredivisionminima\#1}{gregoriotex-signs.tex}
\verb=\gredivisiominima#1=%tex-signs.tex
Macro to typeset a divisio minima.

\begin{argtable}
  \#1 & code & Macros which may happen before the skip but after the divisio minima (typically \verb=\grevepisemus=).\\
\end{argtable}

\macroname{\textbackslash gredivisiominor\#1}{gregoriotex-signs.tex}
\verb=\gredivisiominor#1=%tex-signs.tex
Macro to typeset a divisio minor.

\begin{argtable}
  \#1 & code & Macros which may happen before the skip but after the divisio minor (typically \verb=\grevepisemus=).\\
\end{argtable}

\macroname{\textbackslash gredominica\#1\#2}{gregoriotex-signs.tex}
%\verb=\gredominica#1#2=%tex-signs.tex
Macro to typeset a dominican bar.

\begin{argtable}
  \#1 & integer & Type of dominican bar. 6--11 corresponds to types 1--6.\\
  \#2 & code    & Macros which may happen before the skip but after the divisio minor (typically \verb=\grevepisemus=).\\
\end{argtable}

\macroname{\textbackslash greendnlbarea\#1\#2}{gregoriotex.tex}
%\verb=\greendnlbarea#1#2=%tex.tex
Macro called at beginning of a no line break area.

\begin{argtable}
  \#1 & 0 & Not in the neumes.\\
      & 1 & In the neumes.\\
  \#2 & 0 & Call didn't come from translation centering.\\
      & 1 & Call came from translation centering.\\
\end{argtable}

\macroname{\textbackslash greendofelement\#1\#2}{gregoriotex.tex}
%\verb=\greendofelement#1#2=%tex.tex
Macro to end elements.

\begin{argtable}
  \#1 & 0 & Default space.\\
      & 1 & Larger space.\\
      & 2 & Glyph space.\\
      & 3 & Zero-width space.\\
  \#2 & 0 & Space is breakable.\\
      & 1 & Space is unbreakable.\\
\end{argtable}

\macroname{\textbackslash greendofglyph\#1}{gregoriotex.tex}
%\verb=\greendofglyph#1=%tex.tex
Macro to end a glyph without ending the element.


\begin{argtable}
  \#1 & 0 & Default space.\\
      & 1 & Zero-width space.\\
      & 2 & Space between flat or natural and a note.\\
      & 3 & Space between two puncta inclinata.\\
      & 4 & Space between bivirga or trivirga.\\
      & 5 & space between bistropha or tristropha.\\
      & 6 & Space after a punctum mora XXX: not used yet, not so sure it is a good idea\ldots\\
      & 7 & Space between a punctum inclinatum and a punctum inclinatum debilis.\\
      & 8 & Space between two puncta inclinata debilis.\\
      & 9 & Space before a punctum (or something else) and a punctum inclinatum.\\
      & 10& Space between puncta inclinata (also debilis for now), larger ambitus (range=3rd).\\
      & 11& Space between puncta inclinata (also debilis for now), larger ambitus (range=4th or more).\\
\end{argtable}

\macroname{\textbackslash grefinaldivisionfinalis\#1}{gregoriotex-signs.tex}
%\verb=\grefinaldivisiofinalis#1=%tex-signs.tex
Macro to end a line with a divisio finalis.

\begin{argtable}
  \#1 & 0 & Something does not need to be placed after the divisio finalis.\\
      & 1 & Something needs to be placed after the divisio finalis.\\
\end{argtable}

\macroname{\textbackslash grefinaldivisiomaior\#1}{gregoriotex-signs.tex}
%\verb=\grefinaldivisiomaior#1=%tex-signs.tex
Macro to end a line with a divisio maior.

\begin{argtable}
  \#1 & 0 & Something does not need to be placed after the divisio maior.\\
      & 1 & Something needs to be placed after the divisio maior.\\
\end{argtable}

\macroname{\textbackslash grefirstlinebottomspace\#1\#2}{gregoriotex.tex}
%\verb=\grefirstlinebottomspace#1#2=%tex.tex
Macro for additional bottom space for the first line

\begin{argtable}
  \#1 & 0 & No note below staff.\\
      & 1 & Note below 1st line (position c).\\
      & 2 & Note on 0th line (position b).\\
      & 3 & Note below 0th line (position a).\\
      & 4 & Note below 0th line (position a) with vertical episemus.\\
  \#2 & 0 & No translation in score.\\
      & 1 & Translation present in score.\\
\end{argtable}

\macroname{\textbackslash greflat\#1\#2}{gregoriotex-signs.tex}
%\verb=\greflat#1#2=%tex-signs.tex
Macro to typeset a flat.

\begin{argtable}
  \#1 & character & Height of the flat.\\
  \#2 & 0         & No a flat for a key change.\\
      & 1         & Indicates the a flat for a key change.\\
\end{argtable}

\macroname{\textbackslash greglyph}{gregoriotex-syllable.tex}
%\verb=\greglyph=

\macroname{\textbackslash gregorianmode\#1}{gregoriotex.tex}
%\verb=\gregorianmode#1=%tex.tex
If the gabc file contains a mode in the header, then this function
places said mode as the first (top) annotation.  This function
effectively disables \verb=\setfirstlineaboveinitial=.

\begin{argtable}
  \#1 & 1-8 & Other values are ignored (and \verb=\setfirstlineaboveinitial= should still work).\\
\end{argtable}

Bug: This macro needs to appear before \verb=\greinitial= in the gtex file but \verb=gregoriotex-write.c= places it after.

\macroname{\textbackslash greorioapiversion\#1}{gregoriotex.tex}
%\verb=\gregorioapiversion#1=%tex.tex
Checks to see if GregorioTeX API is version specified by argument (and
therefore compatible with the score.

\begin{argtable}
  \#1 & string & Date in format: yyyymmdd.\\
\end{argtable}

\macroname{\textbackslash grehepisemus\#1\#2\#3\#4}{gregoriotex-signs.tex}
%\verb=\grehepisemus#1#2#3#4=%tex-signs.tex
Macro to typeset a episemus.

\begin{argtable}
  \#1 & character & Height of the episemus.\\
  \#2 & special   & See \nameref{EpisemusSpecial}.\\
  \#3 & integer   & The ambitus of the porrectus or porrectus flexes if the second argument is 9, 10, 11, 21, 22, 23; ignored otherwise..\\
  \#4 & character & Replacement for \#1 if a bridge causes a height substitution.\\
\end{argtable}

\macroname{\textbackslash grehepisemusbottom\#1\#2\#3}{gregoriotex-signs.tex}
%\verb=\grehepisemusbottom#1#2#3=%tex-signs.tex
Macro to typeset a episemus at the bottom of a note

\begin{argtable}
  \#1 & character & Height of the episemus.\\
  \#2 & special   & See \nameref{EpisemusSpecial}.\\
  \#3 & integer   & The ambitus of the porrectus or porrectus flexes if the second argument is 9, 10, 11, 21, 22, 23; ignored otherwise.\\
\end{argtable}

\macroname{\textbackslash grehepisemusbridge\#1\#2\#3}{gregoriotex-signs.tex}
%\verb=\grehepisemusbridge#1#2#3=%tex-signs.tex
Macro to typeset a bridge episemus father the last note of a glyph
(element, syllable) if the next episemus is at the same height.

\begin{argtable}
  \#1 & character & Height of the episemus.\\
  \#2 & special   & See \nameref{EpisemusSpecial}.\\
  \#3 & integer   & The ambitus of the porrectus or porrectus flexes if the second argument is 9, 10, 11, 21, 22, 23; ignored otherwise.\\
\end{argtable}

\macroname{\textbackslash grehighchoralsigh\#1\#2\#3}{gregoriotex-signs.tex}
%\verb=\grehighchoralsign#1#2#3=%tex-signs.tex
Macro for typesetting high choral signs.

\begin{argtable}
  \#1 & character & Height of the sigh.\\
  \#2 & string    & The choral sign.\\
  \#3 & 0 & Choral sign does not occur before last note of podatus, porrectus, or torculus resupinus.\\
      & 1 & Choral sign occurs before last note of podatus, porrectus, or torculus resupinus.\\
\end{argtable}

\macroname{\textbackslash grehyph}{gregoriotex.tex}
%\verb=\grehyph=%tex.tex
Macro used for end of line hyphens.  Defaults to \verb=\grenormalhyph=.

\macroname{\textbackslash greictusa\#1}{gregoriotex-signs.tex}
%\verb=\greictusa#1=%tex-signs.tex
Macro for typesetting an ictus (type a)

\begin{argtable}
  \#1 & integer & Type of glyph the ictus is attached to.  See \nameref{EpisemusSpecial} argument for description of options.\\
\end{argtable}

\macroname{\textbackslash greictust\#1}{gregoriotex-signs.tex}
%\verb=\greictust#1=%tex-signs.tex
Macro for typesetting an ictus (type t)

\begin{argtable}
  \#1 & integer & Type of glyph the ictus is attached to.  See \nameref{EpisemusSpecial} argument for description of options.\\
\end{argtable}

\macroname{\textbackslash grein}{}
%\verb=\grein=

\macroname{\textbackslash gredivisiofinalis\#1}{gregoriotex-signs.tex}
%\verb=\greindivisiofinalis#1=%tex-signs.tex
Macro to typeset a divisio finalis inside a syllable.

\begin{argtable}
  \#1 & code & Macros which may happen before the skip but after the divisio finalis (typically \verb=\grevepisemus=).\\
\end{argtable}

\macroname{\textbackslash greindivisiomaior\#1}{gregoriotex-signs.tex}
%\verb=\greindivisiomaior#1=%tex-signs.tex
Macro to typeset a divisio maior inside a syllable.

\begin{argtable}
  \#1 & code & Macros which may happen before the skip but after the divisio maior (typically \verb=\grevepisemus=).\\
\end{argtable}

\macroname{\textbackslash greindivisiominima\#1}{gregoriotex-signs.tex}
%\verb=\greindivisiominima#1=%tex-signs.tex
Macro to typeset a divisio minima inside a syllable.

\begin{argtable}
  \#1 & code & Macros which may happen before the skip but after the divisio minima (typically \verb=\grevepisemus=).\\
\end{argtable}

\macroname{\textbackslash greindiviciominor\#1}{gregoriotex-signs.tex}
%\verb=\greindivisiominor#1=%tex-signs.tex
Macro to typeset a divisio minor inside a syllable.

\begin{argtable}
  \#1 & code & Macros which may happen before the skip but after the divisio minor (typically \verb=\grevepisemus=).\\
\end{argtable}

\macroname{\textbackslash greindominica\#1\#2}{gregoriotex-signs.tex}
%\verb=\greindominica#1#2=%tex-signs.tex
Macro to typeset a dominican bar inside a syllable.

\begin{argtable}
  \#1 & integer & Type of dominican bar. 6--11 corresponds to types 1--6.\\
  \#2 & code & Macros which may happen before the skip but after the divisio minor (typically \verb=\grevepisemus=).\\
\end{argtable}

\macroname{\textbackslash greinitial\#1}{gregoriotex.tex}
%\verb=\greinitial#1=%tex.tex
Macro to set the initial in the score.

\begin{argtable}
  \#1 & character & The initial letter of the score.\\
\end{argtable}

\macroname{\textbackslash greinsertchiroline}{gregoriotex.tex}
%\verb=\greinsertchiroline=%tex.tex
Macro called at the beginning of a line to insert the chironomic signs.

\macroname{\textbackslash greinvirgula\#1}{gregoriotex-signs.tex}
%\verb=\greinvirgula#1=%tex-signs.tex
Macro to typeset a virgula inside a syllable.

\begin{argtable}
  \#1 & code & Macros which may happen before the skip but after the virgula (typically \verb=\grevepisemus=).\\
\end{argtable}

\macroname{\textbackslash greitalic\#1}{gregoriotex.sty}
%\verb=\greitalic#1=%tex.sty, PlainTeX version in tex.tex
Makes argument (a string) italic.  Accesses \LaTeX\ \verb=\textit= or
Plain\TeX\ \verb=\it= as appropriate.  Corresponds to ``<i></i>'' tags
in gabc.

\begin{argtable}
  \#1 & string & Text to be typeset in italic font.\\
\end{argtable}

\macroname{\textbackslash grelastofline}{gregoriotex.tex}
%\verb=\grelastofline=%tex.tex
Macro to set \verb=\gre@lastoflinecount= to 1 (i.e. mark that this syllable is the last of the line).

\macroname{\textbackslash grelastofscore}{gregoriotex.tex}
%\verb=\grelastofscore=%tex.tex
Macro to mark the syllable as the last of the score.

\macroname{\textbackslash grelinea\#1\#2\#3}{gregoriotex-signs.tex}
%\verb=\grelinea#1#2#3=%tex-signs.tex
Macro for typesetting a linea.

\begin{argtable}
  \#1 & length  & Argument \#2 from greglyph. Height to raise the glyph.\\
  \#2 & length  & Argument \#3 from greglyph. Height of the next note.\\
  \#3 & integer & Argument \#4 from greglyph. The type of glyph.\\
\end{argtable}

\macroname{\textbackslash grelineapunctumcavum}{gregoriotex-signs.tex}
%\verb=\grelineapunctumcavum=
Macro to typeset a linea punctum cavum.

\begin{argtable}
  \#1 & length  & Argument \#2 from greglyph. Height to raise the glyph.\\
  \#2 & length  & Argument \#3 from greglyph. Height of the next note.\\
  \#3 & integer & Argument \#4 from greglyph. The type of glyph.\\
  \#4 & code    & Macros executed before the punctum cavum is written.\\
  \#5 & character & Argument \#5 from greglyph. The signs to typeset before the glyph..\\
\end{argtable}

\macroname{\textbackslash grelowchoralsigh\#1\#2\#3}{gregoriotex-signs.tex}
%\verb=\grelowchoralsign#1#2#3=%tex-signs.tex
Macro for typesetting low choral signs.

\begin{argtable}
  \#1 & character & Height of the sign.\\
  \#2 & string    & The choral sign.\\
  \#3 & 0         & Choral sign does not occur before last note of podatus, porrectus, or torculus resupinus.\\
      & 1         & Choral sign occurs before last note of podatus, porrectus, or torculus resupinus.\\
\end{argtable}

\macroname{\textbackslash grenatural\#1\#2}{gregoriotex-signs.tex}
%\verb=\grenatural#1#2=%tex-signs.tex
Macro to typeset a natural.

\begin{argtable}
  \#1 & character & Height of the flat.\\
  \#2 & 0         & No a flat for a key change.\\
      & 1         & Indicates the a flat for a key change.\\
\end{argtable}

\macroname{\textbackslash grenewline}{gregoriotex.tex}
%\verb=\grenewline=%tex.tex
Macro to call if you want to go to the next line simply.

\macroname{\textbackslash grenewlinewithspace\#1\#2\#3\#4}{gregoriotex.tex}
%\verb=\grenewlinewithspace#1#2#3#4=%tex.tex
Macro called to go to the next line but when there are additional vertical spaces to add

\begin{argtable}
  \#1 & 0 & No note above staff.\\
      & 1 & Note above 4th line (position k).\\
      & 2 & Note on 5th line (position l).\\
      & 3 & Note above 5th line (position m).\\
  \#2 & 0 & No note below staff.\\
      & 1 & Note below 1st line (position c).\\
      & 2 & note on 0th line (position b).\\
      & 3 & Note below 0th line (position a).\\
      & 4 & note below 0th line (position a) with vertical episemus.\\
  \#3 & 0 & No translation.\\
      & 1 & Translation present.\\
  \#4 & 0 & No extra space above staff.\\
      & 1 & Extra space above staff.\\
\end{argtable}

\macroname{\textbackslash grenewparline}{gregoriotex.tex}
%\verb=\grenewparline=%tex.tex
Same as \verb=\grenewline= except line is not justified.

\macroname{\textbackslash grenewparlinewithspace\#1\#2\#3\#4}{gregoriotex.tex}
%\verb=\grenewparlinewithspace#1#2#3#4=%tex.tex
Same as \verb=\grenewlinewithspace= except line is not justified.

\macroname{\textbackslash grenormalhyph}{gregoriotex.tex}
%\verb=\grenormalhyph=%tex.tex, not actually found in gregoriotex-write.c
Macro to typeset a normal hyphen.

\macroname{\textbackslash grenoinitial}{gregoriotex.tex}
%\verb=\grenoinitial=%tex.tex
Macro called when no initial is being set.

\macroname{\textbackslash grenormalinitial}{gregoriotex.tex}
%\verb=\grenormalinitial=%not actually found in gregoriotex-write.c%tex.tex
Macro to cancel a 2-line initial.

\macroname{\textbackslash greoverbrace\#1\#2\#3\#4}{gregoriotex-signs.tex}
%\verb=\greoverbrace#1#2#3#4=%tex-signs.tex
Macro to typeset the curved brace over notes.

\begin{argtable}
  \#1 & dimension & The width.\\
  \#2 & dimension & The vertical shift.\\
  \#3 & dimension & The horizontal shift.\\
  \#4 & 0         & Do not shift to the beginning of the last glyph.\\
      & 1         & Shift to the beginning of the last glyph.\\
\end{argtable}

\macroname{\textbackslash greovercurlybrace\#1\#2\#3\#4\#5}{gregoriotex-signs.tex}
%\verb=\greovercurlybrace#1#2#3#4#5=%tex-signs.tex
Macro to typeset the curly brace over notes.

\begin{argtable}
  \#1 & dimension & The width.\\
  \#2 & 0         & Do not put accent above.\\
      & 1         & Put an accent above.\\
  \#3 & dimension & The vertical shift.\\
  \#4 & dimension & The horizontal shift.\\
  \$5 & 0         & Do not shift to the beginning of the last glyph.\\
      & 1         & Shift to the beginning of the last glyph.\\
\end{argtable}

\macroname{\textbackslash grepunctumcavum\#1\#2\#3\#4\#5}{gregoriotex-signs.tex}
%\verb=\grepunctumcavum#1#2#3#4#5=%tex-signs.tex
Macro to typeset a punctum cavum.

\begin{argtable}
  \#1 & length  & Argument \#2 from greglyph. Height to raise the glyph.\\
  \#2 & length  & Argument \#3 from greglyph. Height of the next note.\\
  \#3 & integer & Argument \#4 from greglyph. The type of glyph.\\
  \#4 & code    & Macros executed before the punctum cavum is written.\\
  \#5 & character & Argument \#5 from greglyph. The signs to typeset before the glyph..\\
\end{argtable}

\macroname{\textbackslash grepunctummora\#1\#2\#3\#4}{gregoriotex-signs.tex}
%\verb=\grepunctummora#1#2#3#4=%tex-signs.tex
Macro for typesetting punctum mora.

\begin{argtable}
  \#1 & character & Height of punctum mora.\\
  \#2 & 1 & Go back to end of punctum.\\
      & 2 & Shift left width of 1 punctum.\\
      & 3 & Shift left width of 1 punctum and ambitus of 1.\\
  \#3 & 0 & Punctum mora does not occur before last note of podatus, porrectus, or torculus resupinus.\\
      & 1 & Punctum mora occurs before last note of podatus, porrectus, or torculus resupinus.\\
  \#4 & 0 & No punctum inclinatum.\\
      & 1 & Punctum inclinatum.\\
\end{argtable}

\macroname{\textbackslash grereversedoccentus\#1\#2}{gregoriotex-signs.tex}
%\verb=\grereversedaccentus#1#2=%tex-signs.tex
Macro for typesetting a reversed accentus.

\begin{argtable}
  \#1 & character & Height of accentus.\\
  \#2 & integer   & Type of glyph the accentus is attached to. See \nameref{EpisemusSpecial} argument for description of options.\\
\end{argtable}

\macroname{\textbackslash grereversedsemicirculus\#1\#2}{gregoriotex-signs.tex}
%\verb=\grereversedsemicirculus#1#2=%tex-signs.tex
Macro for typesetting a reversed semicirculus.

\begin{argtable}
  \#1 & character & Height of semicirculus.\\
  \#2 & integer   & Type of glyph the semicirculus is attached to. See \nameref{EpisemusSpecial} argument for description of options.\\
\end{argtable}

\macroname{\textbackslash grescorereference}{gregoriotex.tex}
%\verb=\grescorereference=%tex.tex
Currently does nothing.

\macroname{\textbackslash gresemicirclus\#1\#2}{gregoriotex-signs.tex}
%\verb=\gresemicirculus#1#2=%tex-signs.tex
Macro for typesetting a semicirculus.

\begin{argtable}
  \#1 & character & Height of semicirculus.\\
  \#2 & integer   & Type of glyph the semicirculus is attached to. See \nameref{EpisemusSpecial} argument for description of options.\\
\end{argtable}

\macroname{\textbackslash gresetbiginitial}{gregoriotex.tex}
%\verb=\gresetbiginitial=%tex.tex
Macro which indicates that a 2-line initial is desired.

\macroname{\textbackslash gresetfixednexttextformat}{gregoriotex-syllable.tex}
%\verb=\gresetfixednexttextformat=

\macroname{\textbackslash gresetfixedtexformat}{gregoriotex-syllable.tex}
%\verb=\gresetfixedtextformat=

\macroname{\textbackslash gresetinitialclef\#1\#2\#3}{gregoriotex-signs.tex}
%\verb=\gresetinitialclef#1#2#3=%tex-signs.tex
Macro for writing initial key.

\begin{argtable}
  \#1 & character & Type of clef (c or f).\\
  \#2 & 1-4       & Line of key.\\
  \#3 & character & Height of flat in key (`a' for no flat).\\
\end{argtable}

\macroname{\textbackslash gresetlinesclef}{gregoriotex.tex}
%\verb=\gresetlinesclef=%#1#2#3#4=%tex.tex
%	possible values for <character>: c, f
%	possible values for <integer2>: 0, 1 (1 = space before clef, 0 = no space)

\macroname{\textbackslash gresettextabovelines\#1}{gregoriotex.tex}
%\verb=\gresettextabovelines#1=
Macro to place argument above the lines and empty\\
\verb=\gre@currenttextabovelines= when done.

\begin{argtable}
  \#1 & string & Text to be placed above the lines.\\
\end{argtable}

\macroname{\textbackslash gresharp\#1\#2}{gregoriotex-signs.tex}
%\verb=\gresharp#1#2=%tex-signs.tex
Macro to typeset a sharp.

\begin{argtable}
  \#1 & character & Height of the flat.\\
  \#2 & 0         & No a flat for a key change.\\
      & 1         & Indicates the a flat for a key change.\\
\end{argtable}


\macroname{\textbackslash gresmallcaps\#1}{gregoriotex.sty}
%\verb=\gresmallcaps#1=%tex.sty, PlainTeX version in tex.tex
Makes argument small capitals. Accesses \LaTeX\ \verb=\textsc= or
Plain\TeX\ \verb=\sc= as appropriate Corresponds to ``<sc></sc>'' tags
in gabc.

\begin{argtable}
  \#1 & string & Text to be typeset in small caps font.\\
\end{argtable}

\macroname{\textbackslash grestar}{gregoriotex-symbol.tex}
%\verb=\grestar=
Macro to typeset an asterisk (*).

\macroname{\textbackslash gresyllable}{gregoriotex-syllable.tex}
%\verb=\gresyllable=

\macroname{\textbackslash gretilde}{gregoriotex.tex}
%\verb=\gretilde=%tex.tex
Macro to print $\sim$.

\macroname{\textbackslash gretranslationcenterend}{gregoriotex.tex}
%\verb=\gretranslationcenterend=%tex.tex
Macro to set \verb=\gre@mustdotranslationcenerend= to 1.

\macroname{\textbackslash grett\#1}{gregoriotex.sty}
%\verb=\grett#1=%tex.sty, PlainTeX version in tex.tex
Makes argument typewriter font.  Accesses \LaTeX\ \verb=\texttt= or
Plain\TeX\ \verb=\tt= as appropriate.

\begin{argtable}
  \#1 & string & Text to typeset in tt font.\\
\end{argtable}

\macroname{\textbackslash greul\#1}{gregoriotex.sty}
%\verb=\greul#1=%tex.sty, PlainTeX version in tex.tex
Makes argument underlined under \LaTeX\ using \verb=\underline=.  Does
nothing in Plain\TeX.

\begin{argtable}
  \#1 & string & Text to typeset underlined.\\
\end{argtable}

\macroname{\textbackslash grevepisemus\#1\#2}{gregoriotex-signs.tex}
%\verb=\grevepisemus#1#2=%tex-signs.tex
Macro for typesetting the vertical episemus.

\begin{argtable}
  \#1 & character & Height of episemus.\\
  \#2 & integer   & Type of glyph the episemus is attached to. See \nameref{EpisemusSpecial} argument for description of options.\\
\end{argtable}

\macroname{\textbackslash grevepisemusictusa\#1\#2}{gregoriotex-signs.tex}
%\verb=\grevepisemusictusa#1#2=%tex-signs.tex
Macro for typesetting the vertical episemus and ictus (type a) on the
same glyph.

\begin{argtable}
  \#1 & character & Height of episemus.\\
  \#2 & integer   & Type of glyph the episemus and ictus are attached to. See \nameref{EpisemusSpecial} argument for description of options.\\
\end{argtable}

\macroname{\textbackslash grevepisemusictust\#1\#2}{gregoriotex-signs.tex}
%\verb=\grevepisemusictust#1#2=%tex-signs.tex
Macro for typesetting the vertical episemus and ictus (type t) on the
same glyph.

\begin{argtable}
  \#1 & character & Height of episemus.\\
  \#2 & integer   & Type of glyph the episemus and ictus are attached to. See \nameref{EpisemusSpecial} argument for description of options.\\
\end{argtable}

\macroname{\textbackslash grevirgula\#1}{gregoriotex-signs.tex}
%\verb=\grevirgula#1=%tex-signs.tex
Macro to typeset a virgula.

\begin{argtable}
  \#1 & code & Macros which may happen before the skip but after the virgula (typically \verb=\grevepisemus=).\\
\end{argtable}

\macroname{\textbackslash grewritetranslation\#1}{gregoriotex.tex}
%\verb=\grewritetranslation#1=%tex.tex
Macro to typeset argument in the translation position.

\begin{argtable}
  \#1 & string & Text to typeset in the translation.\\
\end{argtable}

\macroname{\textbackslash grewritetranslationwithcenterbeginning\#1}{gregoriotex.tex}
%\verb=\grewritetranslationwithcenterbeginning#1=%tex.tex
Macro to typeset argument (a string) in the translation position (at
the beginning of a line?).

\begin{argtable}
  \#1 & string & Text to typeset in the translation (at the beginning of a line).\\
\end{argtable}

\macroname{\textbackslash grezerhyph}{gregoriotex.tex}
%\verb=\grezerhyph=%tex.tex
Macro to typeset a zero-width hyphen (the hyphen is visible, it is
treated as if it had 0 width though.  Used for fine tuning spacing
(especially at line endings).

\macroname{\textbackslash setgregoriofont\#1}{gregoriotex.tex}
%\verb=\setgregoriofont#1=%tex.tex
Macro to set the font used for the glyphs.

\begin{argtable}
  \#1 & string & gregorio; parmesan; greciliae; gregoria\\
\end{argtable}

%%% Local Variables:
%%% mode: latex
%%% TeX-master: "UserManual"
%%% End:
