%GregorioTeX file.
%
% Copyright (C) 2007-2025 The Gregorio Project (see CONTRIBUTORS.md)
%
% This file is part of Gregorio.
%
% Gregorio is free software: you can redistribute it and/or modify
% it under the terms of the GNU General Public License as published by
% the Free Software Foundation, either version 3 of the License, or
% (at your option) any later version.
%
% Gregorio is distributed in the hope that it will be useful,
% but WITHOUT ANY WARRANTY; without even the implied warranty of
% MERCHANTABILITY or FITNESS FOR A PARTICULAR PURPOSE.  See the
% GNU General Public License for more details.
%
% You should have received a copy of the GNU General Public License
% along with Gregorio.  If not, see <http://www.gnu.org/licenses/>.

% this file contains definitions of the symbols

\ifcsname gregoriotex@symbols@loaded\endcsname\endinput\fi%
\def\gregoriotex@symbols@loaded{}%

\gre@declarefileversion{gregoriotex-symbols.tex}{6.1.0}% GREGORIO_VERSION

\RequireLuaModule{gregoriotex}%

\def\gre@symbolfontsize{\f@size}%

% Defines a symbol which automatically scales with \f@size
% #1 = control sequence name
% #2 = font name
% #3 = glyph name or code point
\def\gredefsymbol#1#2#3{%
  \directlua{gregoriotex.init_variant_font([[#2]],false)}%
  \directlua{gregoriotex.def_symbol([[#1]],[[#2]],[[#3]],false)}%
}%

% Defines a symbol which takes a point size as argument
% #1 = control sequence name
% #2 = font name
% #3 = glyph name or code point
\def\gredefsizedsymbol#1#2#3{%
  \directlua{gregoriotex.init_variant_font([[#2]],false)}%
  \directlua{gregoriotex.def_symbol([[#1]],[[#2]],[[#3]],true)}%
}%

%%%%%%%%%%%%%
%% symbols %%
%%%%%%%%%%%%%

\gredefsizedsymbol{greABar}{greextra}{ABar}
\gredefsizedsymbol{greRBar}{greextra}{RBar}
\gredefsizedsymbol{greVBar}{greextra}{VBar}
\gredefsizedsymbol{greABarSlant}{greextra}{ABarSlant}
\gredefsizedsymbol{greRBarSlant}{greextra}{RBarSlant}
\gredefsizedsymbol{greVBarSlant}{greextra}{VBarSlant}
\gredefsizedsymbol{greABarSC}{greextra}{ABarSC}
\gredefsizedsymbol{greRBarSC}{greextra}{RBarSC}
\gredefsizedsymbol{greVBarSC}{greextra}{VBarSC}
\gredefsizedsymbol{greABarSmall}{greextra}{ABarSmall}
\gredefsizedsymbol{greRBarSmall}{greextra}{RBarSmall}
\gredefsizedsymbol{greVBarSmall}{greextra}{VBarSmall}
\gredefsizedsymbol{greABarSmallSlant}{greextra}{ABarSmallSlant}
\gredefsizedsymbol{greRBarSmallSlant}{greextra}{RBarSmallSlant}
\gredefsizedsymbol{greVBarSmallSlant}{greextra}{VBarSmallSlant}
\gredefsizedsymbol{greABarSmallSC}{greextra}{ABarSmallSC}
\gredefsizedsymbol{greRBarSmallSC}{greextra}{RBarSmallSC}
\gredefsizedsymbol{greVBarSmallSC}{greextra}{VBarSmallSC}
\gredefsizedsymbol{greABarCaption}{greextra}{ABarCaption}
\gredefsizedsymbol{greRBarCaption}{greextra}{RBarCaption}
\gredefsizedsymbol{greVBarCaption}{greextra}{VBarCaption}
\gredefsizedsymbol{greABarCaptionSlant}{greextra}{ABarCaptionSlant}
\gredefsizedsymbol{greRBarCaptionSlant}{greextra}{RBarCaptionSlant}
\gredefsizedsymbol{greVBarCaptionSlant}{greextra}{VBarCaptionSlant}
\gredefsizedsymbol{greABarCaptionSC}{greextra}{ABarCaptionSC}
\gredefsizedsymbol{greRBarCaptionSC}{greextra}{RBarCaptionSC}
\gredefsizedsymbol{greVBarCaptionSC}{greextra}{VBarCaptionSC}
\gredefsizedsymbol{greABarAlt}{greextra}{RBar.alt}
\gredefsizedsymbol{greRBarAlt}{greextra}{RBar.alt}
\gredefsizedsymbol{greVBarAlt}{greextra}{VBar.alt}

\def\grebarredsymbol#1#2#3#4#5{%
  \leavevmode\hbox to 0pt{}\kern #4\lower#5%
  \hbox{\csname #2\endcsname{#3}}\kern -#4\relax #1%
}%

%% Intermediate version or barred glyph definition
%% #1: the csname you want to define
%% #2: the letter you'll typeset (you can use bold, italic, etc.)
%% #3: the symbol corresponding to the glyph in greextra. Must be defined with \gredefsizedsymbol
%% #4: the size of greextra
%% #5: the kern of the bar from the beggining of the glyph
%% #6: greextra glyph lowering
\def\gredefbarredsymbol#1#2#3#4#5#6{%
  \expandafter\gdef\csname #1\endcsname {%
    \grebarredsymbol{#2}{#3}{#4}{#5}{#6}%
  }%
}%

%% Most simple version or barred glyph definition
%% #1: A, R or V
%% #2: the negative kern
\def\gresimpledefbarredsymbol#1#2{%
  \gredefbarredsymbol{#1bar}{#1}{gre#1Bar}{\f@size}{#2}{0pt}%
}%

% defining \ABar, \VBar and \RBar for libertine font, define yours for your font
\gresimpledefbarredsymbol{A}{.30em}%
\gresimpledefbarredsymbol{R}{.35em}%
\gresimpledefbarredsymbol{V}{.10em}%

\gre@iflatex{%
  \def\gre@latex@barredsymbol#1#2#3#4{%
    \gre@trace{gre@latex@barredsymbol{#1}{#2}{#3}{#4}}%
    \IfStrEq{\f@series/\f@shape}{#1}{\grebarredsymbol{#2}{#3}{\f@size}{#4}{0pt}}%
    \gre@trace@end%
  }%

  %% define a barred symbol that handles italic, slanted, and small cap shapes
  %% simply (LaTeX-only)
  %% Most simple version or barred glyph definition
  %% #1: A, R or V
  %% #2: the negative kern for upright shape
  %% #3: idem, for italic/slanted shape
  %% #4: idem, for bold shape
  %% #5: idem, for bold and italic/slanted shape
  \def\grelatexsimpledefbarredsymbol#1#2#3#4#5{%
    \expandafter\gdef\csname #1bar\endcsname {%
      \gre@latex@barredsymbol{m/n}{#1}{gre#1Bar}{#2}{%
        \gre@latex@barredsymbol{m/it}{#1}{gre#1BarSlant}{#3}{%
          \gre@latex@barredsymbol{m/sl}{#1}{gre#1BarSlant}{#3}{%
            \gre@latex@barredsymbol{bx/n}{#1}{gre#1Bar}{#4}{%
              \gre@latex@barredsymbol{bx/it}{#1}{gre#1BarSlant}{#5}{%
                \gre@latex@barredsymbol{bx/sl}{#1}{gre#1BarSlant}{#5}{%
                  \gre@latex@barredsymbol{b/n}{#1}{gre#1Bar}{#4}{%
                    \gre@latex@barredsymbol{b/it}{#1}{gre#1BarSlant}{#5}{%
                      \gre@latex@barredsymbol{b/sl}{#1}{gre#1BarSlant}{#5}{%
                      }%
                    }%
                  }%
                }%
              }%
            }%
          }%
        }%
      }%
    }%
  }%

  \grelatexsimpledefbarredsymbol{A}{.30em}{.35em}{.30em}{.33em}%
  \grelatexsimpledefbarredsymbol{R}{.35em}{.38em}{.45em}{.48em}%
  \grelatexsimpledefbarredsymbol{V}{.10em}{.12em}{.14em}{.18em}%
}%

% the gothic R and V
\gredefsymbol{gothRbar}{greextra}{RWithBarGoth}%
\gredefsymbol{gothVbar}{greextra}{VWithBarGoth}%

% the macro to typeset a dagger
\gredefsymbol{GreDagger}{greextra}{Dagger}%

% macro to typeset a (malt) cross
\gredefsymbol{grecross}{greextra}{Cross}%
\gredefsymbol{grealtcross}{greextra}{Cross.alt}%

%stars
\gredefsymbol{greheightstar}{greextra}{StarHeight}%
\gredefsymbol{gresixstar}{greextra}{StarSix}%
\let\GreStar\gresixstar%

%%%%%%%%%%%%
%%  Lines %%
%%%%%%%%%%%%
%Unlike the character symbols above, we require that the font size be specified when using the lines.

\gredefsizedsymbol{greLineOne}{greextra}{Line1}%
\gredefsizedsymbol{greLineTwo}{greextra}{Line2}%
\gredefsizedsymbol{greLineThree}{greextra}{Line3}%
\gredefsizedsymbol{greLineFour}{greextra}{Line4}%
\gredefsizedsymbol{greLineFive}{greextra}{Line5}%

%lines
%#1 is the type of line (1-5)
%#2 is the number of pt of the font it will use
\def\greseparator#1#2{%
  \ifcase#1\relax %
  \or\greLineOne{#2}%
  \or\greLineTwo{#2}%
  \or\greLineThree{#2}%
  \or\greLineFour{#2}%
  \or\greLineFive{#2}%
  \fi%
}%


%%%%%%%%%%%
%% Ornamentation
%%%%%%%%%%%

\gredefsizedsymbol{greOrnamentOne}{greextra}{Drawing1}%
\gredefsizedsymbol{greOrnamentTwo}{greextra}{Drawing2}%

\def\greornamentation#1#2{%
  \ifcase#1\relax%
  \or\greOrnamentOne{#2}%
  \or\greOrnamentTwo{#2}%
  \fi%
}%

%% Specials

\def\gresetspecial#1{\directlua{gregoriotex.define_special_character("#1")}}%
\def\greunsetspecial#1{\directlua{gregoriotex.undefine_special_character("#1")}}%
\def\GreSpecial#1{\directlua{gregoriotex.special_character("#1")}}%

\gresetspecial{A/}{\Abar{}}%
\gresetspecial{\string\037}{\%{}}%
\gresetspecial{R/}{\Rbar{}}%
\gresetspecial{V/}{\Vbar{}}%
\gresetspecial{'ae}{\'\ae{}}%
\gresetspecial{'oe}{\'\oe{}}%
\gresetspecial{ae}{\ae{}}%
\gresetspecial{oe}{\oe{}}%
\gresetspecial{*}{\GreStar{}}%
\gresetspecial{+}{\GreDagger{}}%
\gresetspecial{-}{\gre@char@zerohyph{}}%
\gresetspecial{\string\092}{\textbackslash{}}%
\gresetspecial{\string\038}{\&{}}%
\gresetspecial{\string\035}{\#{}}%
\gresetspecial{\string\095}{\_{}}%
\gresetspecial{\string\126}{\gretilde{}}%
\gre@iflatex{%
  \@ifpackageloaded{luainputenc}{}{%
    \gresetspecial{'æ}{\'\ae{}}%
    \gresetspecial{'œ}{\'\oe{}}%
  }%
}%
\gre@ifnotlatex{%
  \gresetspecial{'æ}{\'\ae{}}%
  \gresetspecial{'œ}{\'\oe{}}%
}%
